\documentclass[a4paper,14pt,russian]{extarticle}

\usepackage{extsizes}
\usepackage[T2A]{fontenc}
\usepackage[utf8]{inputenc}
\usepackage[russian]{babel}
\usepackage{pscyr} % Русские шрифты
\usepackage{indentfirst} % отделять первую строку раздела абзацным отступом тоже
\renewcommand{\rmdefault}{ftm} % Times New Roman

\usepackage[usenames,dvipsnames]{color} % названия цветов

\usepackage{algorithm}
\usepackage{algorithmic}
\usepackage{syntax}
\usepackage{listings}
\lstset{language=C++,
	basicstyle=\ttfamily\small,
	keywordstyle=\color{blue}\ttfamily,
	stringstyle=\color{red}\ttfamily,
	commentstyle=\color{green}\ttfamily,
	morecomment=[l][\color{magenta}]{\#}
}

\usepackage{titlesec}

\titleformat{\section}
{\normalsize\bfseries}
{\thesection}
{1em}{}

\titleformat{\subsection}
{\normalsize\bfseries}
{\thesubsection}
{1em}{}


\usepackage{tocloft}
\renewcommand{\cfttoctitlefont}{\hspace{0.38\textwidth} \bfseries\MakeUppercase}
\renewcommand{\cftbeforetoctitleskip}{1em}
\renewcommand{\cftaftertoctitle}{\mbox{}\hfill \\ \mbox{}\hfill{\footnotesize Стр.}\vspace{0em}}
\renewcommand{\cftsecfont}{\hspace{31pt}}
\renewcommand{\cftsubsecfont}{\hspace{11pt}}
\renewcommand{\cftparskip}{-1mm}
\renewcommand{\cftdotsep}{1}
\renewcommand{\cftsecleader}{\cftdotfill{\cftdotsep}} 
\setcounter{tocdepth}{2} % задать глубину оглавления — до subsection включительно

\usepackage{adjustbox}

\usepackage[tableposition=top]{caption}
\usepackage{subcaption}
\DeclareCaptionLabelFormat{gostfigure}{Рисунок #2}
\DeclareCaptionLabelFormat{gosttable}{Таблица #2}
\DeclareCaptionLabelSeparator{gost}{~---~}
\captionsetup{labelsep=gost}
\captionsetup[figure]{labelformat=gostfigure}
\captionsetup[table]{labelformat=gosttable,justification=justified, singlelinecheck=off}
\renewcommand{\thesubfigure}{\asbuk{subfigure}}

\usepackage{needspace}

\usepackage{svg}

\begin{document}
\thispagestyle{empty}
\begin{center}
	{\large
		Университет ИТМО \\
		Факультет программной инженерии и компьютерной техники
	}
\end{center}
\vspace{\stretch{2}}
\begin{center}
	{\large
		Параллельные вычисления\\
	}
	\it{
		Лабораторная работа №3
	}
\end{center}
\vspace{\stretch{6}}

\begin{table}[h]
	\begin{flushright}
		\begin{tabular}{l}
			Выполнил:\\ 
			студент группы P42111\\ 
			\it Губарев В.Ю.
		\end{tabular} 
	\end{flushright}	
\end{table}

\vspace{\stretch{4}}
\begin{center}
	2019, г. Санкт-Петербург
\end{center}
\newpage

\tableofcontents

\newpage


\section{Задание}


На компьютере с многоядерным процессором установить ОС Linux
и компилятор GCC версии не ниже 4.7.2. При невозможности установить Linux или отсутствии компьютера с многоядерным процессором можно выполнять лабораторную работу на виртуальной машине.

Скомпилировать написанную программу, используя встроенное в
gcc средство автоматического распараллеливания Graphite с помощью следующей команды “/home/user/gcc -O2 -Wall -Werror -floopparallelize-all -ftree-parallelize-loops=K lab1.c -o lab1-par-K” (переменной K поочерёдно присвоить хотя бы 4 различных целых значений, выбор обосновать).\\

Индивидуальное задание:
\begin{table}[h]
	\begin{tabular}{|l|l|l|}
		\hline
		Губарев&Владимир&Юрьевич \\\hline
		7&8&7 \\\hline
	\end{tabular}
\end{table}

\begin{table}[h]
	\begin{tabular}{|l|l|l|}
		\hline
		A &	392&	 \\\hline
		B&	вариант&	Задание\\\hline
		7&	3&	Гиперболический тангенс с последующим уменьшением на 1\\\hline
		8&	1&	Модуль синуса (т.е. M2[i] = |sin(M2[i] + M[i1])|)\\\hline
		6&	5&	Выбор меньшего\\\hline
		7&	3&	Пирамидальная сортировка (сортировка кучи, Heapsort)\\\hline
	\end{tabular}
\end{table}


\section{Аппаратная и программная конфигурация}


Процессор:
\begin{lstlisting}
CPU(s):                40
On-line CPU(s) list:   0-19
Off-line CPU(s) list:  20-39
Thread(s) per core:    1
Core(s) per socket:    10
Socket(s):             2
NUMA node(s):          2
Vendor ID:             GenuineIntel
CPU family:            6
Model:                 63
Model name:            Intel(R) Xeon(R) CPU E5-2676v3 @ 2.40GHz
\end{lstlisting}

RAM:
\begin{verbatim}
160GiB DDR4
\end{verbatim}

Использованные компиляторы:
\begin{lstlisting}
[17/09/2019-10:36:16] ~/university/parallel/lab1/project
vladimirg@ip-10-230-0-33|tb> /opt/gcc-8.2.0/bin/gcc --version
gcc (GCC) 8.2.0

[17/09/2019-10:36:22] ~/university/parallel/lab1/project
vladimirg@ip-10-230-0-33|tb> /opt/llvm-8/bin/clang --version
clang version 8.0.0 (tags/RELEASE_800/final)

[17/09/2019-12:36:22] ~/
./intel/system_studio_2019/bin/icc --version
icc (ICC) 19.0.4.235 20190416
Copyright (C) 1985-2019 Intel Corporation.  All rights reserved.
\end{lstlisting}


\section{Исходный код}


\begin{lstlisting}
#include <float.h>
#include <math.h>
#include <stdio.h>
#include <stdlib.h>
#include <sys/time.h>

const static int c_a = 392;
const static int c_experiments = 50;

void generate(unsigned int seed, double * p, unsigned int N, unsigned int min, unsigned int max)
{
    unsigned int i;
    for (i = 0; i < N; i++) {
        p[i] = (rand_r(&seed) % max) + min;
    }
}

void lab_swap(double * lhs, double * rhs)
{
    double tmp = *lhs;
    *lhs = *rhs;
    *rhs = tmp;
}

// To heapify a subtree rooted with node i which is 
// an index in arr[]. n is size of heap 
void heapify(double * arr, unsigned int n, unsigned int i)
{ 
    unsigned int largest = i; // Initialize largest as root
    unsigned int l = 2*i + 1; // left = 2*i + 1
    unsigned int r = 2*i + 2; // right = 2*i + 2
  
    // If left child is larger than root 
    if (l < n && arr[l] > arr[largest]) {
        largest = l;
    }

    // If right child is larger than largest so far 
    if (r < n && arr[r] > arr[largest]) {
        largest = r;
    }

    // If largest is not root 
    if (largest != i) {
        lab_swap(&arr[i], &arr[largest]);
        heapify(arr, n, largest);
    } 
} 
  
void heapsort(double * arr, int n) 
{ 
    // Build heap (rearrange array)
    int i;
    for (i = n / 2 - 1; i >= 0; i--) {
        heapify(arr, n, i);
    }

    // One by one extract an element from heap
    for (i=n-1; i>=0; i--) { 
        // Move current root to end
        lab_swap(&arr[0], &arr[i]);
        // call max heapify on the reduced heap
        heapify(arr, i, 0);
    }
}

double lab_abs(double v)
{
    if (v < 0) {
        return -v;
    }
    return v;
}

double lab_min(double lhs, double rhs)
{
    return lhs > rhs ? rhs : lhs;
}

void print_array(double * p, unsigned int N)
{
    unsigned int i = 0;
    for (i = 0; i < N - 1; i++) {
        printf("%f ", p[i]);
    }
    printf("%f\n", p[N - 1]);
}

int main(int argc, char * argv[])
{
    if (argc < 2) {
        return -1;
    }

    struct timeval begin, end;
    const int N = atoi(argv[1]);
    if (N < 0) {
        return -2;
    }

    double * m1 = malloc(sizeof(double) * N);
    double * m2 = malloc(sizeof(double) * N / 2);

    gettimeofday(&begin, NULL);
    double reduced_sum = 0.0;
    unsigned int i;
    for (i = 0; i < c_experiments; i++) {
        // 1. Generate: M1 of N elements, M2 of N/2 elements
        generate(i, m1, N, 1, c_a);
        //    puts("M1");
        //    print_array(m1, N);
        generate(i, m2, N / 2, c_a, 10 * c_a);
        //    puts("M2");
        //    print_array(m2, N / 2);
        // 2. Map: tanh(M1[j]) - 1 , j e N ; M2[j] = abs(sin(M2[j] + M2[j-1])) , j e N/2
        unsigned int j;
        for (j = 0; j < N; j++) {
            m1[j] = tanh(m1[j]) - 1;
        }
        //    puts("M1 tanh");
        //    print_array(m1, N);
        m2[0] = lab_abs(sin(m2[0] /* + 0.0 */));
        for (j = 1; j < N / 2; j++) {
            m2[j] = lab_abs(sin(m2[j] + m2[j - 1]));
        }
        //    puts("M2 abs sin sum");
        //    print_array(m2, N / 2);
        // 3. Merge: M2[j] = min(M1[j], M2[j]) , j e N/2
        for (j = 0; j < N / 2; j++) {
            m2[j] = lab_min(m1[j], m2[j]);
        }
        //    puts("min of M1 M2");
        //    print_array(m2, N / 2);
        // 4. Sort: heapsort(M2, N/2)
        heapsort(m2, N/2);
        //    puts("sorted");
        //    print_array(m2, N / 2);
        // 5. Reduce: 1. min_non_zero(M2)
        //            2. if (((long)(M2[i] / min_non_zero)) & ~(1))
        //                   sum += sin(M2[i])
        double min_non_zero = DBL_MAX;
        for (j = 0; j < N / 2; j++) {
            if (m2[j] != 0) {
                min_non_zero = lab_min(min_non_zero, m2[j]);
            }
        }
        //    printf("Min non zero: %f\n", min_non_zero);
        for (j = 0; j < N / 2; j++) {
            if (((long)(m2[i] / min_non_zero)) & ~(1)) {
                reduced_sum += sin(m2[j]);
            }
        }
        //    printf("Sum: %e\n", reduced_sum);
    }
    gettimeofday(&end, NULL);
    long delta_ms = 1000 * (end.tv_sec - begin.tv_sec) + (end.tv_usec - begin.tv_usec) / 1000;
    printf("N = %d. Milliseconds passed: %ld\n", N, delta_ms);
    printf("N = %d. X=%e\n", N, reduced_sum / c_experiments);

    return 0;
}

\end{lstlisting}

\newpage


\section{Результаты проведения экспериментов}

\subsection{GCC}

Использованные флаги компиляции:
\begin{lstlisting}
-O3 -floop-parallelize-all -ftree-parallelize-loops=${THREADS}
\end{lstlisting}


\begin{table}[h]
	\caption{Последовательное выполнение}
	\label{gcc-seq}
	\begin{adjustbox}{width=\textwidth,nofloat=table}
			\begin{tabular}{|l|l|l|l|l|l|l|l|l|l|l|l|}
				\hline
				& N1   & N1+1  & N1+2   & N1+3   & N1+4   & N1+5   & N1+6   & N1+7   & N1+8   & N1+9   & N2     \\\hline
				N      & 1750 & 71775 & 141800 & 211825 & 281850 & 351875 & 421900 & 491925 & 561950 & 631975 & 702000 \\\hline
				seq(N), мс & 10   & 236   & 435    & 628    & 823    & 1008   & 1210   & 1397   & 1598   & 1791   & 2003   \\\hline
				X      & 0    & 0     & 0      & 0      & 0      & 0      & 0      & 0      & 0      & 0      & 0 \\\hline
			\end{tabular}
	\end{adjustbox}
\end{table}

\begin{table}[h]
	\caption{Выполнение при 2 потоках}
	\label{gcc-2}
	\begin{adjustbox}{width=\textwidth,nofloat=table}
	\begin{tabular}{|l|l|l|l|l|l|l|l|l|l|l|l|}
		\hline
		& N1   & N1+1  & N1+2   & N1+3   & N1+4   & N1+5   & N1+6   & N1+7   & N1+8   & N1+9   & N2     \\\hline
		N        & 1750        & 71775       & 141800      & 211825      & 281850      & 351875     & 421900      & 491925      & 561950      & 631975      & 702000      \\ \hline
		par-2(N) & 9           & 224         & 374         & 572         & 759         & 941        & 1105        & 1310        & 1475        & 1702        & 1896        \\ \hline
		X        & 0           & 0           & 0           & 0           & 0           & 0          & 0           & 0           & 0           & 0           & 0           \\ \hline
		S(2)  & 1,111 & 1,054 & 1,163  & 1,098  & 1,084  & 1,071  & 1,095  & 1,066  & 1,083  & 1,052  & 1,056  \\ \hline
	\end{tabular}
	\end{adjustbox}
\end{table}

\begin{table}[h]
	\caption{Выполнение при 4 потоках}
	\label{gcc-4}
	\begin{adjustbox}{width=\textwidth,nofloat=table}
		\begin{tabular}{|l|l|l|l|l|l|l|l|l|l|l|l|}
			\hline
			         & N1    & N1+d  & N1+d   & N1+d   & N1+d   & N1+d   & N1+d   & N1+d   & N1+d   & N1+d   & N2     \\ \hline
			N        & 1750  & 71775 & 141800 & 211825 & 281850 & 351875 & 421900 & 491925 & 561950 & 631975 & 702000 \\ \hline
			par-4(N) & 9     & 217   & 366    & 559    & 731    & 913    & 1082   & 1309   & 1445   & 1611   & 1790   \\ \hline
			X        & 0     & 0     & 0      & 0      & 0      & 0      & 0      & 0      & 0      & 0      & 0      \\ \hline
			S(4)  & 1,111 & 1,088 & 1,189  & 1,123  & 1,126  & 1,104  & 1,118  & 1,067  & 1,106  & 1,112  & 1,119  \\ \hline
		\end{tabular}
	\end{adjustbox}
\end{table}

\begin{table}[h]
	\caption{Выполнение при 8 потоках}
	\label{gcc-8}
	\begin{adjustbox}{width=\textwidth,nofloat=table}
		\begin{tabular}{|l|l|l|l|l|l|l|l|l|l|l|l|}
			\hline
	         & N1   & N1+d  & N1+d   & N1+d   & N1+d   & N1+d   & N1+d   & N1+d   & N1+d   & N1+d   & N2     \\ \hline
	N        & 1750 & 71775 & 141800 & 211825 & 281850 & 351875 & 421900 & 491925 & 561950 & 631975 & 702000 \\ \hline
	par-8(N) & 10   & 217   & 363    & 538    & 715    & 908    & 1069   & 1242   & 1427   & 1595   & 1787   \\ \hline
	X        & 0    & 0     & 0      & 0      & 0      & 0      & 0      & 0      & 0      & 0      & 0      \\ \hline
	speedup  & 1    & 1,088 & 1,198  & 1,167  & 1,151  & 1,11   & 1,132  & 1,125  & 1,12   & 1,123  & 1,121  \\ \hline
		\end{tabular}
	\end{adjustbox}
\end{table}


\begin{table}[h]
	\caption{Выполнение при 16 потоках}
	\label{gcc-16}
	\begin{adjustbox}{width=\textwidth,nofloat=table}
		\begin{tabular}{|l|l|l|l|l|l|l|l|l|l|l|l|}
			\hline
			& N1   & N1+d  & N1+d   & N1+d   & N1+d   & N1+d   & N1+d   & N1+d   & N1+d   & N1+d   & N2     \\ \hline
			N         & 1750 & 71775 & 141800 & 211825 & 281850 & 351875 & 421900 & 491925 & 561950 & 631975 & 702000 \\ \hline
			par-16(N) & 10   & 219   & 360    & 535    & 711    & 927    & 1062   & 1276   & 1410   & 1594   & 1766   \\ \hline
			X         & 0    & 0     & 0      & 0      & 0      & 0      & 0      & 0      & 0      & 0      & 0      \\ \hline
			S(16)   & 1    & 1,078 & 1,208  & 1,174  & 1,158  & 1,087  & 1,139  & 1,095  & 1,133  & 1,124  & 1,134  \\ \hline
		\end{tabular}
	\end{adjustbox}
\end{table}


\begin{figure}[H]
	\centering
	\includegraphics[width=\textwidth]{gcc}
	\caption{Время выполнения полезной нагрузки программы от N, программы, скомпилированной GCC}
	\label{pic:gcc}
\end{figure}


\subsection{Clang}

Выполнение настоящего пункта требует сборки и установки LLVM Polly.\\

Использованные флаги компиляции:
\begin{lstlisting}
-O3 -mllvm -polly -mllvm -polly-parallel -mllvm
-polly-num-threads=${THREADS}
\end{lstlisting}

\begin{table}[H]
	\caption{Последовательное выполнение}
	\label{clang-seq}
	\begin{adjustbox}{width=\textwidth,nofloat=table}
		\begin{tabular}{|l|l|l|l|l|l|l|l|l|l|l|l|}
			\hline
			& N1   & N1+1  & N1+2   & N1+3   & N1+4   & N1+5   & N1+6   & N1+7   & N1+8   & N1+9   & N2     \\\hline
N       & 1750  & 71775 & 141800 & 211825 & 281850 & 351875 & 421900 & 491925 & 561950 & 631975 & 702000 \\ \hline
seq(N)  & 10    & 234   & 420    & 573    & 764    & 1003   & 1190   & 1336   & 1557   & 1770   & 1924   \\ \hline
X       & 0     & 0     & 0      & 0      & 0      & 0      & 0      & 0      & 0      & 0      & 0      \\ \hline
		\end{tabular}
	\end{adjustbox}
\end{table}

\begin{table}[H]
	\caption{Выполнение при 2 потоках}
	\label{clang-2}
	\begin{adjustbox}{width=\textwidth,nofloat=table}
		\begin{tabular}{|l|l|l|l|l|l|l|l|l|l|l|l|}
			\hline
			& N1   & N1+1  & N1+2   & N1+3   & N1+4   & N1+5   & N1+6   & N1+7   & N1+8   & N1+9   & N2     \\\hline
N        & 1750  & 71775 & 141800 & 211825 & 281850 & 351875 & 421900 & 491925 & 561950 & 631975 & 702000 \\ \hline
par-2(N) & 9     & 233   & 422    & 572    & 806    & 998    & 1194   & 1376   & 1570   & 1769   & 1952   \\ \hline
X        & 0     & 0     & 0      & 0      & 0      & 0      & 0      & 0      & 0      & 0      & 0      \\ \hline
speedup  & 1,111 & 1,004 & 0,995  & 1,002  & 0,948  & 1,005  & 0,997  & 0,971  & 0,992  & 1,001  & 0,986  \\ \hline
		\end{tabular}
	\end{adjustbox}
\end{table}

\begin{table}[H]
	\caption{Выполнение при 4 потоках}
	\label{clang-4}
	\begin{adjustbox}{width=\textwidth,nofloat=table}
		\begin{tabular}{|l|l|l|l|l|l|l|l|l|l|l|l|}
			\hline
			& N1    & N1+d  & N1+d   & N1+d   & N1+d   & N1+d   & N1+d   & N1+d   & N1+d   & N1+d   & N2     \\ \hline
N        & 1750  & 71775 & 141800 & 211825 & 281850 & 351875 & 421900 & 491925 & 561950 & 631975 & 702000 \\ \hline
par-4(N) & 9     & 230   & 427    & 571    & 806    & 994    & 1184   & 1364   & 1558   & 1749   & 1964   \\ \hline
X        & 0     & 0     & 0      & 0      & 0      & 0      & 0      & 0      & 0      & 0      & 0      \\ \hline
speedup  & 1,111 & 1,017 & 0,984  & 1,004  & 0,948  & 1,009  & 1,005  & 0,979  & 0,999  & 1,012  & 0,98   \\ \hline
		\end{tabular}
	\end{adjustbox}
\end{table}

\begin{table}[H]
	\caption{Выполнение при 8 потоках}
	\label{clang-8}
	\begin{adjustbox}{width=\textwidth,nofloat=table}
		\begin{tabular}{|l|l|l|l|l|l|l|l|l|l|l|l|}
			\hline
			& N1   & N1+d  & N1+d   & N1+d   & N1+d   & N1+d   & N1+d   & N1+d   & N1+d   & N1+d   & N2     \\ \hline
N        & 1750  & 71775 & 141800 & 211825 & 281850 & 351875 & 421900 & 491925 & 561950 & 631975 & 702000 \\ \hline
par-8(N) & 9     & 231   & 427    & 569    & 760    & 951    & 1188   & 1377   & 1570   & 1766   & 1947   \\ \hline
X        & 0     & 0     & 0      & 0      & 0      & 0      & 0      & 0      & 0      & 0      & 0      \\ \hline
speedup  & 1,111 & 1,013 & 0,984  & 1,007  & 1,005  & 1,055  & 1,002  & 0,97   & 0,992  & 1,002  & 0,988  \\ \hline
		\end{tabular}
	\end{adjustbox}
\end{table}

\begin{table}[H]
	\caption{Выполнение при 16 потоках}
	\label{clang-16}
	\begin{adjustbox}{width=\textwidth,nofloat=table}
		\begin{tabular}{|l|l|l|l|l|l|l|l|l|l|l|l|}
			\hline
			& N1   & N1+d  & N1+d   & N1+d   & N1+d   & N1+d   & N1+d   & N1+d   & N1+d   & N1+d   & N2     \\ \hline
N         & 1750  & 71775 & 141800 & 211825 & 281850 & 351875 & 421900 & 491925 & 561950 & 631975 & 702000 \\ \hline
par-16(N) & 9     & 231   & 428    & 570    & 799    & 996    & 1208   & 1374   & 1574   & 1764   & 1955   \\ \hline
X         & 0     & 0     & 0      & 0      & 0      & 0      & 0      & 0      & 0      & 0      & 0      \\ \hline
speedup   & 1,111 & 1,013 & 0,981  & 1,005  & 0,956  & 1,007  & 0,985  & 0,972  & 0,989  & 1,003  & 0,984  \\ \hline
		\end{tabular}
	\end{adjustbox}
\end{table}

\begin{figure}[H]
	\centering
	\includegraphics[width=\textwidth]{clang}
	\caption{Время выполнения полезной нагрузки программы от N, программы, скомпилированной Clang}
	\label{pic:clang}
\end{figure}


\subsection{ICC}

Использованные флаги компиляции:
\begin{lstlisting}
-O3 -parallel -par-threshold=${THREADS}
\end{lstlisting}


\begin{table}[H]
	\caption{Последовательное выполнение}
	\label{icc-seq}
	\begin{adjustbox}{width=\textwidth,nofloat=table}
		\begin{tabular}{|l|l|l|l|l|l|l|l|l|l|l|l|}
			\hline
			& N1   & N1+1  & N1+2   & N1+3   & N1+4   & N1+5   & N1+6   & N1+7   & N1+8   & N1+9   & N2     \\\hline
N       & 1750 & 71775 & 141800 & 211825 & 281850 & 351875 & 421900 & 491925 & 561950 & 631975 & 702000 \\ \hline
seq(N)  & 7    & 206   & 377    & 545    & 724    & 890    & 1063   & 1192   & 1416   & 1589   & 1765   \\ \hline
X       & 0    & 0     & 0      & 0      & 0      & 0      & 0      & 0      & 0      & 0      & 0      \\ \hline
speedup & 1    & 1,078 & 1,208  & 1,174  & 1,158  & 1,087  & 1,139  & 1,095  & 1,133  & 1,124  & 1,134  \\ \hline
		\end{tabular}
	\end{adjustbox}
\end{table}

\begin{table}[H]
	\caption{Выполнение при 2 потоках}
	\label{icc-2}
	\begin{adjustbox}{width=\textwidth,nofloat=table}
		\begin{tabular}{|l|l|l|l|l|l|l|l|l|l|l|l|}
			\hline
			& N1   & N1+1  & N1+2   & N1+3   & N1+4   & N1+5   & N1+6   & N1+7   & N1+8   & N1+9   & N2     \\\hline
N        & 1750  & 71775 & 141800 & 211825 & 281850 & 351875 & 421900 & 491925 & 561950 & 631975 & 702000 \\ \hline
par-2(N) & 44    & 184   & 540    & 446    & 604    & 751    & 901    & 1096   & 1248   & 1419   & 1602   \\ \hline
X        & 0     & 0     & 0      & 0      & 0      & 0      & 0      & 0      & 0      & 0      & 0      \\ \hline
speedup  & 0,159 & 1,12  & 0,698  & 1,222  & 1,199  & 1,185  & 1,18   & 1,088  & 1,135  & 1,12   & 1,102  \\ \hline
		\end{tabular}
	\end{adjustbox}
\end{table}

\begin{table}[H]
	\caption{Выполнение при 4 потоках}
	\label{icc-4}
	\begin{adjustbox}{width=\textwidth,nofloat=table}
		\begin{tabular}{|l|l|l|l|l|l|l|l|l|l|l|l|}
			\hline
			& N1    & N1+d  & N1+d   & N1+d   & N1+d   & N1+d   & N1+d   & N1+d   & N1+d   & N1+d   & N2     \\ \hline
N        & 1750  & 71775 & 141800 & 211825 & 281850 & 351875 & 421900 & 491925 & 561950 & 631975 & 702000 \\ \hline
par-4(N) & 16    & 187   & 295    & 450    & 603    & 753    & 900    & 1101   & 1257   & 1395   & 1552   \\ \hline
X        & 0     & 0     & 0      & 0      & 0      & 0      & 0      & 0      & 0      & 0      & 0      \\ \hline
speedup  & 0,438 & 1,102 & 1,278  & 1,211  & 1,201  & 1,182  & 1,181  & 1,083  & 1,126  & 1,139  & 1,137  \\ \hline
		\end{tabular}
	\end{adjustbox}
\end{table}

\begin{table}[H]
	\caption{Выполнение при 8 потоках}
	\label{icc-8}
	\begin{adjustbox}{width=\textwidth,nofloat=table}
		\begin{tabular}{|l|l|l|l|l|l|l|l|l|l|l|l|}
			\hline
			& N1   & N1+d  & N1+d   & N1+d   & N1+d   & N1+d   & N1+d   & N1+d   & N1+d   & N1+d   & N2     \\ \hline
N        & 1750  & 71775 & 141800 & 211825 & 281850 & 351875 & 421900 & 491925 & 561950 & 631975 & 702000 \\ \hline
par-8(N) & 16    & 178   & 298    & 453    & 597    & 765    & 911    & 1093   & 1230   & 1407   & 1582   \\ \hline
X        & 0     & 0     & 0      & 0      & 0      & 0      & 0      & 0      & 0      & 0      & 0      \\ \hline
speedup  & 0,438 & 1,157 & 1,265  & 1,203  & 1,213  & 1,163  & 1,167  & 1,091  & 1,151  & 1,129  & 1,116  \\ \hline
		\end{tabular}
	\end{adjustbox}
\end{table}

\begin{table}[H]
	\caption{Выполнение при 16 потоках}
	\label{icc-16}
	\begin{adjustbox}{width=\textwidth,nofloat=table}
		\begin{tabular}{|l|l|l|l|l|l|l|l|l|l|l|l|}
			\hline
			& N1   & N1+d  & N1+d   & N1+d   & N1+d   & N1+d   & N1+d   & N1+d   & N1+d   & N1+d   & N2     \\ \hline
N         & 1750  & 71775 & 141800 & 211825 & 281850 & 351875 & 421900 & 491925 & 561950 & 631975 & 702000 \\ \hline
par-16(N) & 18    & 193   & 296    & 458    & 600    & 768    & 910    & 1082   & 1258   & 1397   & 1578   \\ \hline
X         & 0     & 0     & 0      & 0      & 0      & 0      & 0      & 0      & 0      & 0      & 0      \\ \hline
speedup   & 0,389 & 1,067 & 1,274  & 1,19   & 1,207  & 1,159  & 1,168  & 1,102  & 1,126  & 1,137  & 1,119  \\ \hline
		\end{tabular}
	\end{adjustbox}
\end{table}

\begin{figure}[H]
	\centering
	\includegraphics[width=\textwidth]{icc}
	\caption{Время выполнения полезной нагрузки программы от N программы, скомпилированной ICC}
	\label{pic:icc}
\end{figure}


\section{Выводы о проведенных экспериментах}

На основании проведенных экспериментов можно сделать следующие выводы:
\begin{enumerate}
	\item GCC и ICC показывают существенный эффект от использования средств автоматического распараллеливания программы (в пределах 15\%).
	\item ICC показывает значительное преимущество над GCC по времени выполнения полезной нагрузки программы во всех исследованных конфигурациях, включая однопоточное исполнение заданного алгоритма.
	\item Единственным исключением из пункта выше можно назвать выполнение программы при N=N1. При этих входных данных GCC показывает себя лучше ICC. Наиболее разумным объяснением такого поведения может являться момент, в который создаются потоки для последующего параллельного выполнения цикла. А именно, делается ли это в начале работы программы (до взятия времени начала работы алгоритма) или после. Во втором случае в результирующее время будет включены временные затраты на создание потоков.
	\item Согласно отладочной информации процесса автоматического распараллеливания GCC, эффект был достигнут за счет распараллеливания цикла выбора минимального значения среди массива M2 и первой половины массива M1.
	\item Связка инструментальных средств Clang + Polly + LLVM не показала какого- либо эффекта от своего использования. Согласно отладочной информации процесса автоматического распараллеливания, Polly не удалось найти ни одного подходящего для этих улучшений цикла.
	\item Согласно strace, ICC создавал не более 3 потоков дополнительно (+1 главный поток), в то время, как общее число потоков в GCC было равно запрашиваемому (2, 4, 8, 16, соответственно).
\end{enumerate}



\end{document}
